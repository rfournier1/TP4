\documentclass[a4paper]{article}

\usepackage[T1]{fontenc}
\usepackage[utf8]{inputenc}
\usepackage{lmodern}
\usepackage{microtype}
\usepackage[french]{babel}
\usepackage[top=3cm,bottom=3cm,headheight=60pt,footskip=20pt]{geometry}
\usepackage{hyperref}
\usepackage{fancyhdr}
\usepackage{tikz}
\usepackage{pgfkeys}
\usepackage{pgfopts}
\usepackage{afterpage}
\usepackage{authoraftertitle}
\usepackage{enumitem}
\usepackage{multicol}
\usepackage{listings}

\setlength{\columnseprule}{0pt}

\usetikzlibrary{positioning,patterns,snakes,matrix}
%\usetikzlibrary{matrix}

%\geometry{showframe=true}

\pagestyle{fancy}
\lhead{}
\rhead{\includegraphics[scale=0.3]{logoINSA.png}}
\lfoot{}
\cfoot{\vspace{0.7cm}
\thepage}
\rfoot{}

\renewcommand{\headrulewidth}{0.4pt}
\renewcommand{\footrulewidth}{0.4pt}

\title{TP C++ \no 4 : 
\bigbreak
Analyse de logs apache
}
\bigbreak
\author{DURAND Bastien \and FOURNIER Romain}

\afterpage{
\chead{\textsc{TP C++ \no 4\\
\nouppercase{\leftmark}}}
\cfoot{\hspace{0.5cm}DURAND Bastien\hspace{1cm} FOURNIER Romain\\
\vspace{0.5cm}
\thepage}}
%\lhead{\MakeLowercase{\leftmark}}}

\begin{document}

\maketitle
\thispagestyle{fancy}

\tableofcontents
\smallbreak

\section*{Introduction}\noindent
\newpage

\section{Cahier des charges et spécifications}\noindent
	
	Nous avons créé un programme d'exploitation de fichiers log qui respecte les besoins furmulés dans l'énoncé :
	-par défaut, l'outil affiche dans la console sous forme textuelle la liste des 10 documents les plus populaires par ordre décroissants. L'outils s'appelle en ligne de commande par la commande : ./analog [options] nomfichier.log.
	-l'outil peut prendre 3 options en ligne de commande :
	\begin{itemize}
		\item{} [-g nomfichier.dot] produit un fichier au format Graphviz.
		\item{} [-e] permet d'exclure les ressources qui ont une extension js, image ou css.
		\item{} [-t] permet de ne prendre en compte que les hits dans un créneau horaire correspondant à l'interval [heure , heure+1[.
	\end{itemize}
	
	L'expression initiale des besoins étant un peu floue, nous y avons ajouté quelques spécifications :
	\begin{itemize}
		\item Par défaut, l'outil affiche dans la console sous forme textuelle la liste des 10 documents les plus populaires par ordre décroissants.
					\begin{itemize}
							\item Si il y a moins de 10 documents consultés, l'outil donnera la liste de tous les documents classés par popularité.
							\item Si la recherche n'a retourné aucun documents, l'outil donnera une réponse prédéfinie.
							\item Si le fichier .log est vide, inexistant, mal formé ou non spécifié, l'outil affiche l'aide dans le terminal.
					\end{itemize}
		\item{} [-g nomfichier.dot] produit un fichier au format Graphviz. 
					\begin{itemize}
							\item Pour les ressource n'ayant pas de referer (accès direct à la ressource), un noeud '' - '' représente les accès directe dans le graphe.
							\item Si le nom du fichier .dot n'est pas spécifié, l'outil affiche l'aide dans le terminal.
							\item Si un fichier du même nom existe déjà, une confirmation est demandée à l'utilisateur avant d'écraser le fichier.
					\end{itemize}
		\item{} [-e] permet d'exclure les ressources qui ont une extension js, image ou css.
					\begin{itemize}
							\item Si une ressource n'a pas d'extension, elle sera séléctionnée.
					\end{itemize}
		\item{} [-t] permet de ne prendre en compte que les hits dans un créneau horaire correspondant à l'interval [heure, heure+1[.
					\begin{itemize}
							\item L'outil devra aussi fonctionner si la plage horaire s'étend sur 2 jours.
							\item Si l'heure n'est pas spécifiée ou au mauvais format, l'outil affiche l'aide dans le terminal.
					\end{itemize}
	\end{itemize}
							

\section{Conception}\noindent

	Notre application ne comporte qu'une seule interface et une seule réalisation associée. En effet, nous avons décider d'utiliser une structure de donnée de la STL pour gérer nos données. Nous n'avons donc pas eu à implémenter de classes supplémentaires.

\newpage
\section{Données}\noindent

Les données tirées de l'exploitation des logs seront stockées dans plusieurs conteneurs de la stl de type "unordered\_map". A la différence d'une "map" ce conteneur ne stocke pas ses clées triées. Ainsi ce n'est pas sous forme d'arbre binaire rouge et noir qu'une "unordered\_map" est implémentée mais sous la forme d'une table de hachage. Le cout en insertion ou recherche est en O(1) en moyenne et O(n) dans le pire des cas. 
\\
Les données extraites des logs sont des ressources et des referers. Voici la manières dont nous les stockons :
\begin{itemize}
\item chaque ressource est une clé dans une "unordered\_map" "générale",
\item clé étant associée à une valeur de type "unordered\_map",
\item les referers pointant sur une ressource sont clés de l'"unordered\_map" liée à cette ressource,
\item clé étant associée à une valeur de type int,
\item cet entier désigant le nombre de fois que la ressource a été demandée via ce referer.
\end{itemize}

\begin{tikzpicture}[draw, minimum width=1cm, minimum height=0.5cm]
    	
    	\node[text width=3cm] at (-4,1.5) {keys};
    
    \matrix (ref1) at (-5,0) [matrix of nodes, nodes={draw, nodes={draw}}, nodes in empty cells, minimum width=3cm]
    {
       http://ref\_1\\ http://ref\_2\\ ...\\ http://ref\_n\\
    };
    
    \draw[red,thick] (-2,0) circle (0.5) ;
    \node[text width=3cm] at (-1.5,1) {hash function};
    
    \node[draw] (nb11) at (0,1) {3};
    \node[draw] (nb12) at (0,0) {1};
    \node[draw] (nb13) at (0,-1) {1};
  
    \node[text width=3cm] at (1,1.5) {values};
    
	\draw[-latex] (ref1-1-1.east) .. controls (-2,0) .. (nb11.west);
	\draw[-latex] (ref1-2-1.east) .. controls (-2,0) .. (nb13.west);    
	\draw[-latex] (ref1-4-1.east) .. controls (-2,0) .. (nb12.west);     
    
    
    	\node[text width=3cm] at (-4,-2.5) {keys};
    	
    \matrix (ref2) at (-5,-4) [matrix of nodes, nodes={draw, nodes={draw}}, nodes in empty cells, minimum width=3cm]
    {
       http://ref\_1\\ http://ref\_2\\ ...\\ http://ref\_n\\
    };
    
    \draw[red,thick] (-2,-4) circle (0.5) ;
    \node[text width=3cm] at (-1.5,-3) {hash function};
    
    \node[draw] (nb21) at (0,-3) {10};
    \node[draw] (nb22) at (0,-4) {1};
    \node[draw] (nb23) at (0,-5) {7};
  
    \node[text width=3cm] at (1,-2.5) {values};
    
	\draw[-latex] (ref2-1-1.east) .. controls (-2,-4) .. (nb23.west);
	\draw[-latex] (ref2-2-1.east) .. controls (-2,-4) .. (nb21.west);    
	\draw[-latex] (ref2-4-1.east) .. controls (-2,-4) .. (nb22.west);
	
	\node[text width=3cm] at (-4,-6.5) {keys};	
	
	\matrix (ref3) at (-5,-8) [matrix of nodes, nodes={draw, nodes={draw}}, nodes in empty cells, minimum width=3cm]
    {
       http://ref\_1\\ http://ref\_2\\ ...\\ http://ref\_n\\
    };
    
    \draw[red,thick] (-2,-8) circle (0.5) ;
    \node[text width=3cm] at (-1.5,-7) {hash function};
    
    \node[draw] (nb31) at (0,-7) {5};
    \node[draw] (nb32) at (0,-8) {4};
    \node[draw] (nb33) at (0,-9) {1};
  
    \node[text width=3cm] at (1,-6.5) {values};
    
	\draw[-latex] (ref3-1-1.east) .. controls (-2,-8) .. (nb32.west);
	\draw[-latex] (ref3-2-1.east) .. controls (-2,-8) .. (nb33.west);    
	\draw[-latex] (ref3-4-1.east) .. controls (-2,-8) .. (nb31.west);
	
	\node[text width=3cm, thick, font=\fontsize{15}{0}\selectfont, thick] (resk) at (-12,3) {keys};
	\node[text width=3cm, thick, font=\fontsize{15}{0}\selectfont, thick] (resv) at (-2,3) {values};
	
	\matrix (res) at (-13,-4) [matrix of nodes, nodes={draw, nodes={draw}}, nodes in empty cells, minimum width=3cm]
    {
       /ressource\_1\\ /ressource\_2\\ ...\\ /ressource\_n\\
    };
    
    
    \draw[red,thick] (-9,-4) circle (1);
    \node[text width=3cm, thick, font=\fontsize{15}{0}\selectfont, thick] at (-9,-2.5) {hash function};
    
    \draw[-latex] (res-1-1.east) .. controls (-9,-4) .. (ref2-1-1.north west);
	\draw[-latex] (res-2-1.east) .. controls (-9,-4) .. (ref1-1-1.north west);    
	\draw[-latex] (res-4-1.east) .. controls (-9,-4) .. (ref3-1-1.north west);
	
	\draw[thick,decoration={brace,raise=0.5cm, mirror},decorate] (ref3.south west) -- (nb33.south east);
	\draw[thick,decoration={brace,raise=0.5cm},decorate] (resk.north west) -- (resv.north east);
	
	\node[text width=3cm, thick, font=\fontsize{15}{0}\selectfont, thick] at (-7,4.5) {unordered\_map};
	\node[text width=3cm] (resv) at (-3,-10.5) {unordered\_map};
	
\end{tikzpicture}

\clearpage
\section{Réalisation}\noindent

\end{document}